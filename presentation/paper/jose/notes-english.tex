\documentclass[11pt]{article}

% Use the following to compile
% mkdir tmp
% pdflatex -aux-directory=tmp -output-directory=tmp --shell-escape notes.tex


% Package use definitions
\usepackage[margin=1in]{geometry}
\usepackage{fancyhdr}
\usepackage[parfill]{parskip}
\usepackage{graphicx}
\usepackage{comment}
\usepackage[outputdir=tmp]{minted}
\usepackage[dvipsnames]{xcolor}
\usepackage{listings}
\usepackage[hidelinks]{hyperref}

% Header and footer setup
\pagestyle{fancy}
\rhead{Notes}
\lhead{Book Title}
\renewcommand{\headrulewidth}{1pt}
\renewcommand{\footrulewidth}{1pt}

% Image directory specification
\graphicspath{ {./images/} }

% Settings minted option for the entire document
\definecolor{LightGray}{rgb}{0.9, 0.9, 0.9}
\setminted{frame=lines,framesep=2mm,bgcolor=LightGray,linenos,
  fontsize=\footnotesize, baselinestretch=1.2}

% Start of document
\begin{document}

% Title page and table of contents setup
\begin{titlepage}
  \begin{center}
    \vspace*{1cm}
    \Huge
    \textbf{APPERO}\\
    \vspace{0.4cm}
    Optimisation Hivernale\\
    \vspace{5cm}
    \textbf{Jose A. Henriquez Roa}
    \newpage
    \normalsize
    \tableofcontents
    \newpage
  \end{center}
\end{titlepage}

% Document Body:
\section{Contributions}
Hoping to do the project on my own I chose an empty group on the 8'th of August.
I started doing the research to solve the problem that same day. By the time the
annoucement was made that groups with less than 3 or 4 members were merged on
the 17'th of August I already had a working implementation of an algorithm to
solve the problem.\\\\
During our first group meeting, the decision was then made to split the
implementation and presentation of the three main algorithms I used in between
the three of us. Their new implementations were then to be integrated later on.
However, as time passed the decision was made to keep my implementation
integrated into the main algorithm. This is the reason why they are all listed
here.
\subsection{Main algorithm implementation}
\textbf{Path:} \texttt{appero/snowymontreal/solve.py}\\\\
This is the algorithm that presents the solution to the main
\textbf{Route Inspection Problem} or \textbf{Chinese Postman Problem}.
\subsection{Main algorithm presentation}
\textbf{Path:} \texttt{appero/videos/pb-intro.mp4}\\\\
I was responsible for the presentation of this one. In the final video this one
is under the \textit{Introduction} section. The singled out clip can also be
found in the above mentioned \textbf{Path}.
\subsection{Hungarian algorithm implementation}
\textbf{Path:} \texttt{appero/snowymontreal/hungarian.py}\\\\
This algorithm is used to find the minimum matching in the bipartite graph built
from the nodes of degree bigger and smaller than 0 (more information on this on
the presentation video). This algorithm was developed and published by Harold
Kuhn in 1955.
\subsection{Hungarian algorithm presentation}
\textbf{Path:} \texttt{appero/videos/hungarian.mp4}\\\\
I was also responsible for the presentation of this algorithm. This one is found
on the respective \textit{Hungarian Algorithm} section of the final video or in
the aforementioned \textbf{Path}.
\subsection{Floyd-Warshall algorithm implementation}
\textbf{Path:} \texttt{appero/snowymontreal/floyd\_warshall.py}\\\\
This algorithm is used to find the minimum weight path in between all nodes in
the graph. It is mostly used for the path duplication section of the main
algorithm (more information on this on the presentation video). This algorithm
was developed and published by Rober Floyd and Stephen Warshall in 1962.
\subsection{Hierholzer algorithm implementation}
\textbf{Path:} \texttt{appero/snowymontreal/hierholzer.py}\\\\
After path duplication this algorithm is used for finding the final returned
eulerian path (more information on this on the presentation video). This
algorithm was developed and published by Carl Hierholzer in 1873.
\subsection{Test suit}
\textbf{Path: \texttt{appero/test.py}}\\\\
This is a simple black box test suit for verifying the main algorithms output
correctfulness.
\section{Learnings}
\subsection{Chinese postman problem}
As of now I have learned of a way to solve the Chinese Postman Problem.
\subsection{Hungarian Algorithm}
I have also gotten more familiar with the inner workings of the Hungarian
matching algorithm.
\end{document}
