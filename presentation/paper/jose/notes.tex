\documentclass[11pt]{article}

% Use the following to compile
% mkdir tmp
% pdflatex -aux-directory=tmp -output-directory=tmp --shell-escape notes.tex


% Package use definitions
\usepackage[margin=1in]{geometry}
\usepackage{fancyhdr}
\usepackage[parfill]{parskip}
\usepackage{graphicx}
\usepackage{comment}
\usepackage[outputdir=tmp]{minted}
\usepackage[dvipsnames]{xcolor}
\usepackage{listings}
\usepackage[hidelinks]{hyperref}

% Header and footer setup
\pagestyle{fancy}
\rhead{Jose A. Henriquez Roa}
\lhead{APPERO}
\renewcommand{\headrulewidth}{1pt}
\renewcommand{\footrulewidth}{1pt}

% Image directory specification
\graphicspath{ {./images/} }

% Settings minted option for the entire document
\definecolor{LightGray}{rgb}{0.9, 0.9, 0.9}
\setminted{frame=lines,framesep=2mm,bgcolor=LightGray,linenos,
  fontsize=\footnotesize, baselinestretch=1.2}

% Start of document
\begin{document}

% Title page and table of contents setup
\begin{titlepage}
  \begin{center}
    \vspace*{1cm}
    \Huge
    \textbf{APPERO}\\
    \vspace{0.4cm}
    Optimisation Hivernale\\
    \vspace{5cm}
    \textbf{Jose A. Henriquez Roa}
    \newpage
    \normalsize
    \tableofcontents
    \newpage
  \end{center}
\end{titlepage}

% Document Body:
\section{Contributions}
Dans l'espoir de faire le projet par moi-même, j'ai choisi un groupe vide le 8
août. J'ai commencé à faire des recherches pour résoudre le problème le même
jour. Au moment où l'annonce a été faite que les groupes de moins de 3 ou 4
membres ont été fusionnés le 17 août, j'avais déjà une implémentation
fonctionnelle d'un algorithme pour résoudre le problème.\\\\
Lors de notre première réunion de groupe, la décision a ensuite été prise de
répartir la mise en œuvre et la présentation des trois principaux algorithmes
que j'ai utilisés entre nous trois. Leurs nouvelles implémentations devaient
ensuite être intégrées ultérieurement. Cependant, au fil du temps, la décision a
été prise de garder mon implémentation intégrée dans l'algorithme
principal. C'est la raison pour laquelle ils sont tous répertoriés ici.
\subsection{Main algorithm implementation}
\textbf{Path:} \texttt{appero/snowymontreal/solve.py}\\\\
C'est l'algorithme qui présente la solution aux principaux
Problème: \textbf{Route Inspection Problem} ou \textbf{Chinese Postman Problem}.
\subsection{Main algorithm présentation}
\textbf{Path:} \texttt{appero/videos/pb-intro.mp4}\\\\
J'étais responsable de la présentation de celui-ci. Dans la vidéo finale,
celle-ci se trouve dans la section \textit{Introduction}. Le clip peut également
être trouvé dans le \textit{Path} mentionné ci-dessus.
\subsection{Hungarian algorithm implementation}
\textbf{Path:} \texttt{appero/snowymontreal/hungarian.py}\\\\
Cet algorithme est utilisé pour trouver la couplage minimale dans le
graphe bipartite construit à partir des sommet de degré plus grand et plus petit
que 0 (plus d'informations à ce sujet sur la vidéo de présentation). Cet
algorithme a été développé et publié par Harold Kuhn en 1955.
\subsection{Hungarian algorithm présentation}
\textbf{Path:} \texttt{appero/videos/hungarian.mp4}\\\\
J'étais également responsable de la présentation de cet algorithme. Celui-ci se
trouve dans la section respective \textit{Hungarian Algorithm} de la vidéo
finale ou dans le \textit{Path} mentionné ci-dessus.
\subsection{Floyd-Warshall algorithm implementation}
\textbf{Path:} \texttt{appero/snowymontreal/floyd\_warshall.py}\\\\
Cet algorithme est utilisé pour trouver le chemin de poids minimal entre tous
les sommet du graph. Il est principalement utilisé pour la section de
duplication de chemin de l'algorithme principal (plus d'informations à ce sujet
sur la vidéo de présentation). Cet algorithme a été développé et publié par
Rober Floyd et Stephen Warshall en 1962.
\subsection{Hierholzer algorithm implementation}
\textbf{Path:} \texttt{appero/snowymontreal/hierholzer.py}\\\\
Après duplication de chemin, cet algorithme est utilisé pour trouver le chemin
eulérien final retourné (plus d'informations à ce sujet sur la vidéo de
présentation). Cet algorithme a été développé et publié par Carl Hierholzer en
1873.Après duplication de chemin, cet algorithme est utilisé pour trouver le
chemin eulérien final retourné (plus d'informations à ce sujet sur la vidéo de
présentation). Cet algorithme a été développé et publié par Carl Hierholzer en
1873.
\subsection{Test suit}
\textbf{Path: \texttt{appero/test.py}}\\\\
Il s'agit d'une simple "black box testsuit" pour vérifier
la sortie de l'algorithm principal.
\section{Éléments appris}
\subsection{Chinese postman problem}
J'ai appris un moyen de résoudre le problème du facteur chinois.
\subsection{Hungarian Algorithm}
Je me suis également familiarisé avec le fonctionnement interne du
"Hungarian Algorithm".
\end{document}
